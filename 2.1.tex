\begin{document}


\section{\texorpdfstring{1.1栅格化 \(+\)
园林分区}{1.1栅格化 + 园林分区}}\label{ux6805ux683cux5316-ux56edux6797ux5206ux533a}

由于用excel数据画图有些湖泊、山体应当闭合而未闭合,所以先进行数据处理。把未闭合的水域闭合,未闭合的山体闭合。非闭合山体用
\(1000\mathrm{mm}\) 带宽缓冲。

这样可以使得植物、山石、水域都是一个闭合的区域,能够被多边形填充。然后再往每个斑块中填充元素点,每个点代表
\(10000\mathrm{m}^2\)
。如果同一景观元素重叠(比如最常见的植被),那么重点的地方每个单位只填充一次,避免重复考虑。

然后用栅格法使得园林分块化:

并计算每个栅格水体、植物和假山的比例。

\section{1.栅格法(Grid-basedrepresentation)}\label{ux6805ux683cux6cd5grid-basedrepresentation}

假设自然元素的坐标集合为

\[
\mathcal{D} = \{(x_i,y_i,t_i)\}_{i = 1}^N,t_i\in \{水体,植物,假山\}
\]

我们把园林的平面区域划分为一个 \(G\times G\) 的栅格,每个点
\((x_{i},y_{i})\) 被映射到一个栅格坐标:

\[
g_{x}(i) = \left\lfloor \frac{x_{i} - x_{\min}}{x_{\max} - x_{\min}}\cdot (G - 1)\right\rfloor
\]

\[
g_{y}(i) = \left\lfloor \frac{y_{i} - y_{\min}}{y_{\max} - y_{\min}}\cdot (G - 1)\right\rfloor
\]

然后统计每个格子 \((g_{x},g_{y})\) 内,不同类型元素的数量:

\[
n_{g,t} = \# \{i\mid g_x(i) = g_x,g_y(i) = g_y,t_i = t\}
\]

并计算比例:

\[
r_{g,t} = \frac{n_{g,t}}{\sum_{t'}n_{g,t'}}\quad (t\in \{水体,植物,假山\})
\]

此外,还可以定义栅格中心点坐标:

\[
(x_{g},y_{g}) = \left(x_{\min} + \left(g_{x} + \frac{1}{2}\right)\frac{x_{\max} - x_{\min}}{G},y_{\min} + \left(g_{y} + \frac{1}{2}\right)\frac{y_{\max} - y_{\min}}{G}\right)
\]

于是,每个栅格可表示为一个特征向量:

\[
\mathbf{f}_g = \left(r_{g,\mathcal{N}\{\mathcal{N}\}},r_{g,\mathcal{H}\{\mathcal{H}\}},r_{g,\mathcal{H}\{\mathcal{H}\}},x_g,y_g\right)
\]

然后再用K- means方法,对栅格进行聚类:

\section{2.K-means聚类}\label{k-meansux805aux7c7b}

K- means的目标是把所有栅格特征 \(\mathbf{f}_g\) 分成 \(k\)
个簇。其优化目标为:

\[
\min_{\{C_j\} ,\{\mu_j\}}\sum_{j = 1}^{k}\sum_{\mathbf{f}_g\in C_j}\| \mathbf{f}_g - \mu_j\| ^2
\]

其中:

\(C_j =\) 第 \(j\) 个簇(cluster) \(\mu_j =\) 第 \(j\)
个簇的质心(meanvector)

选代过程:

\begin{enumerate}
\def\labelenumi{\arabic{enumi}.}
\tightlist
\item
  初始化簇中心 \(\mu_{j}\)\\
\item
  分配:每个 \(\mathbf{f}_g\) 分到最近的 \(\mu_{j}\)\\
\item
  更新:对每个簇,重新计算质心
\end{enumerate}

\[
\mu_{j} = \frac{1}{|C_{j}|}\sum_{\mathbf{f}_{g}\in C_{j}}\mathbf{f}_{g}
\]

循环直到收敛。

同时,K- means聚类的优化目标其实就是SSE,为每一个聚类数 \(\mathrm{k}\)
计算一个SSE,当SSE的减小率达到 \(15\%\) 时,即

\[
r(K) = \frac{SSE(K - 1) - SSE(K)}{SSE(K - 1)}
\]

则 \(\mathrm{k}\) 为最终选择的聚类数。

这样,也就把每个园林分成了 \(\mathrm{k}\) 个分区。

\section{1.2接下来计算第一问的元素特征}\label{ux63a5ux4e0bux6765ux8ba1ux7b97ux7b2cux4e00ux95eeux7684ux5143ux7d20ux7279ux5f81}

1.2.1对每一个园林,先计算各元素的比例

在每一个分区中,计算:

\section{5.分区比例(加权平均)}\label{ux5206ux533aux6bd4ux4f8bux52a0ux6743ux5e73ux5747}

在分区 \(C_m\) 内,按栅格点数 \(T_{g}\) 加权:

\[
R_{m,t} = \frac{\sum_{g\in C_m}r_{g,t}T_g}{\sum_{g\in C_m}T_g},\quad t\in \{水体,植物,假山\}
\]

权重是每个栅格内的元素点数量,避免有些栅格只有少数点,并且某一元

素的比例更大,最终使得最终结果偏差放大。

再以每个分区的栅格数量占比为权重,计算整个园的比例:

\section{6. 全园比例}\label{ux5168ux56edux6bd4ux4f8b}

再按每个分区的栅格数量 \(|C_{m}|\) 加权:

\[
R_{t} = \frac{\sum_{m = 1}^{K}R_{m,t}|C_{m}|}{\sum_{m = 1}^{K}|C_{m}|},\quad t\in \{\text{水体,植物,假山}\}
\]

这三个指标代表了园林三种自然要素比例对审美的影响。

\section{1.2.2
然后计算园林中,各个元素的离散程度}\label{ux7136ux540eux8ba1ux7b97ux56edux6797ux4e2dux5404ux4e2aux5143ux7d20ux7684ux79bbux6563ux7a0bux5ea6}

利用每一个分区的各元素比例,还有计算出的全园各元素的平均比例,利用方差公式:

\[
s_{t}^{2} = \frac{1}{K - 1}\sum_{m = 1}^{K}\left(R_{m,t} - \mu_{t}\right)^{2}
\]

衡量不同分区不同景观元素的变化程度。就可以得到\ldots{}

\section{1.2.3
计算整个园林的元素多样性}\label{ux8ba1ux7b97ux6574ux4e2aux56edux6797ux7684ux5143ux7d20ux591aux6837ux6027}

对于每个分区,先用下式计算整个分区的香农指数:

\[
H_{m} = -\sum_{t\in \{\text{水体,植物,假山}\}}R_{m,t}\ln R_{m,t}
\]

指数越大代表每个分区的景观多样性越丰富,反而越单一。

然后再用每个分区的栅格数量占比为权重,用下式衡量整个园区的景观多样性:

\[
H_{\text{全园}} = \frac{\sum_{m = 1}^{K}w_{m}\cdot H_{m}}{\sum_{m = 1}^{K}w_{m}}
\]

\section{1.2.4
还有个主成分分析}\label{ux8fd8ux6709ux4e2aux4e3bux6210ux5206ux5206ux6790}

\end{document}
